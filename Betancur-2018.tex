\documentclass{beamer}
\usepackage[no-math]{fontspec}
\usepackage{xeCJK}
\setCJKmainfont{Source Han Sans TW}
\hypersetup{colorlinks,linkcolor=}

\usetheme{CambridgeUS}
\title[Deep learning (Betancur \textit{et al}, 2018)]{Deep learning for prediction of obstructive disease from fast myocardial perfusion SPECT: a multicenter study}
\subtitle{Betancur \textit{et al}, \textit{J Am Coll Cardiol Img}, 2018}
\author[Chen-Pang He]{何震邦 (Chen-Pang He), Intern}
\date{June 15, 2018}
\institute[CGH]{Cathay General Hospital}

\newcommand*{\solo}[1]{\includegraphics[width=\textwidth, height=0.8\textheight, keepaspectratio]{#1}}

\begin{document}
\maketitle

\section{Summary}
\subsection{Abstract}
\begin{frame}{Abstract}
    \begin{description}
        \item[Objectives] 
            The study evaluated the automatic prediction of obstructive disease
            from myocardial perfusion imaging (MPI) by deep learning (DL) as
            compared with total perfusion deficit (TPD).
        \item[Background]
            Deep convolutional neural networks (CNN) trained with a large
            multicenter population may provide improved prediction of
            per-patient and per-vessel coronary artery disease (CAD) from
            single-photon emission computed tomography (SPECT) MPI.
    \end{description}
\end{frame}

\section{Methods}
\subsection{Study population}
\begin{frame}{Study population}
    \begin{itemize}
        \item Totally 1638 patients (67\% men) referred for SPECT MPI
            \begin{itemize}
                \item From 2008 to 2015
                \item At 9 national and international sites (Canada, Switzerland, Israel)
            \end{itemize}
        \item Without prior
            \begin{itemize}
                \item Myocardial infarction (MI)
                \item Percutaneous coronary intervention (PCI)
                \item Coronary artery bypass graft surgery (CABG)
            \end{itemize}
        \item Underwent a clinically indicated invasive coronary angiography
            (ICA) within 180 days of MPI.
    \end{itemize}
\end{frame}

\begin{frame}{Demography}
    \begin{center}
        \solo{0.eps}
    \end{center}
\end{frame}

\subsection{Image acquisition}
\begin{frame}{Image acquisition}
    Stress MPI with high-efficiency solid-state SPECT scanners
    \begin{itemize}
        \item $^{99\textup{m}}$Tc-sestamibi: 1469 patients (90\%)
        \item $^{99\textup{m}}$Tc-tetrofosmin: 169 patients (10\%)
    \end{itemize}
\end{frame}

\begin{frame}{Stress and rest images}
    Among the study patients
    \begin{itemize}
        \item 1526 (93\%) underwent same-day stress and rest image acquisition
        \item 65 (4\%) had stress and rest images acquired in different days
        \item 47 (3\%) had only stress images.
    \end{itemize}

    The mean weight-adjusted stress dose was $673 \pm 458$ MBq ($18.2 \pm 12.4$ mCi).
\end{frame}

\begin{frame}{Type of stress}
    \begin{itemize}
        \item Treadmill exercise testing: 639 (39\%)
        \item Pharmacologic stress: 999 (61\%)
    \end{itemize}

    Upright and supine stress imaging began 15 to 60 min after stress, and
    lasted 4 to 6 min.
\end{frame}

\subsection{Invasive coronary angiography}
\begin{frame}{Invasive coronary angiography}
    \begin{itemize}
        \item ICA was performed within 6 months of the MPI examination.
        \item All angiograms were interpreted by an on-site cardiologist.
    \end{itemize}
\end{frame}

\begin{frame}{Gold standard for obstructive CAD}
    Luminal diameter narrowing in any of
    \begin{itemize}
        \item $\ge 50\%$ of the left main artery
        \item $\ge 70\%$ in any of
            \begin{itemize}
                \item Left anterior descending artery (LAD)
                \item Left circumflex artery (LCx)
                \item Right coronary atery (RCA)
            \end{itemize}
    \end{itemize}
\end{frame}

\subsection{Image processing and QPS}
\begin{frame}{Image processing}
    Left ventricular (LV) myocardial contours were computed using standard QPS
    (Quantitative Perfusion SPECT) software and verified by a nuclear medicine
    technologist blinded to any clinical findings.
\end{frame}

\begin{frame}{Automated SPECT MPI quantification}
    \begin{description}
        \item[Blackout polar map]
            Blacking out raw polar map samples below the abnormality threshold
            of 2.5 standard deviations.
        \item[TPD polar map]
            The ratio of the summed normalized severities divided by the total
            theoretic maximum TPD.
    \end{description}
\end{frame}

\subsection{Deep learning model}
\begin{frame}{DL prediction of obstructive CAD from MPI}
    \begin{center}
        \solo{1.eps}
    \end{center}
\end{frame}

\begin{frame}{How machine learning works}
    \href{https://brohrer.github.io/}{Brandon Rohrer} made clear explanations
    with no fancy math and no computer jargon.
    \begin{itemize}
        \item \href{https://brohrer.github.io/deep_learning_demystified.html}{Deep learning demystified}
        \item \href{https://brohrer.github.io/how_convolutional_neural_networks_work.html}{How convolutional neural networks work}
    \end{itemize}
\end{frame}

\begin{frame}{Prediction of per-vessel/per-patient disease}
    \begin{itemize}
        \item
            DL computes a probability of obstructive CAD in each vessel without
            predefined subdivision of the polar map.
        \item
            The average error between the predicted per-vessel probabilities
            and the disease location as defined by invasive angiography.
        \item
            Multivessel disease prediction is based on the patterns for each
            vessel.
    \end{itemize}
\end{frame}

\section{Results}
\begin{frame}{Abbreviations}
    \begin{description}
        \item[AUC] area under curve
        \item[CI] confidence interval
    \end{description}
\end{frame}

\begin{frame}{DL prediction of obstructive disease}
    \begin{center}
        \solo{2.eps}
    \end{center}
\end{frame}

\begin{frame}{Per-vessel analysis}
    \begin{center}
        \solo{3.eps}
    \end{center}
\end{frame}

\begin{frame}{Case example}
    \begin{center}
        \solo{4.eps}
    \end{center}
\end{frame}
\end{document}
